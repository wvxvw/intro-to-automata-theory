% Created 2015-09-13 Sun 14:32
\documentclass[11pt]{article}
\usepackage[utf8]{inputenc}
\usepackage[T1]{fontenc}
\usepackage{fixltx2e}
\usepackage{graphicx}
\usepackage{longtable}
\usepackage{float}
\usepackage{wrapfig}
\usepackage{rotating}
\usepackage[normalem]{ulem}
\usepackage{amsmath}
\usepackage{textcomp}
\usepackage{marvosym}
\usepackage{wasysym}
\usepackage{amssymb}
\usepackage{capt-of}
\usepackage{hyperref}
\tolerance=1000
\usepackage[utf8]{inputenc}
\usepackage{commath}
\usepackage{pgf}
\usepackage{tikz}
\usetikzlibrary{shapes,backgrounds}
\usepackage{marginnote}
\usepackage{listings}
\usepackage{enumerate}
\usepackage{algpseudocode}
\usepackage{algorithm}
\usepackage{mathtools}
\usetikzlibrary{arrows,automata}
\setlength{\parskip}{16pt plus 2pt minus 2pt}
\renewcommand{\arraystretch}{1.6}
\DeclareMathOperator{\Neg}{Neg}
\author{Oleg Sivokon}
\date{\textit{<2015-09-11 Fri>}}
\title{Assignment 13, Authomata Theory}
\hypersetup{
 pdfauthor={Oleg Sivokon},
 pdftitle={Assignment 13, Authomata Theory},
 pdfkeywords={Automata Theory, Formal Languages, Assignment},
 pdfsubject={Third assignment in the course 20440 Automata and Formal Languages},
 pdfcreator={Emacs 25.0.50.1 (Org mode 8.3beta)}, 
 pdflang={English}}
\begin{document}

\maketitle
\tableofcontents

\definecolor{codebg}{rgb}{0.96,0.99,0.8}
\definecolor{codestr}{rgb}{0.46,0.09,0.2}
\lstset{%
  backgroundcolor=\color{codebg},
  basicstyle=\ttfamily\scriptsize,
  breakatwhitespace=false,
  breaklines=false,
  captionpos=b,
  framexleftmargin=10pt,
  xleftmargin=10pt,
  framerule=0pt,
  frame=tb,
  keepspaces=true,
  keywordstyle=\color{blue},
  showspaces=false,
  showstringspaces=false,
  showtabs=false,
  stringstyle=\color{codestr},
  tabsize=2
}
\lstnewenvironment{maxima}{%
  \lstset{%
    backgroundcolor=\color{codebg},
    escapeinside={(*@}{@*)},
    aboveskip=20pt,
    captionpos=b,
    label=,
    caption=,
    showstringspaces=false,
    frame=single,
    framerule=0pt,
    basicstyle=\ttfamily\scriptsize,
    columns=fixed}}{}
}
\makeatletter
\newcommand{\verbatimfont}[1]{\renewcommand{\verbatim@font}{\ttfamily#1}}
\makeatother
\verbatimfont{\small}%
\clearpage

\section{Problems}
\label{sec:orgheadline13}

\subsection{Problem 1}
\label{sec:orgheadline2}
Prove that
\begin{align*}
  L = \{w \in \{a, b, c, d\}^* \;|\; w=dv, v \in \{a, b, c\}^*,
  \#_a(w) \cdot \#_c(w) < \#_b(w) \}
\end{align*}

is not regular.

\subsubsection{Answer 1}
\label{sec:orgheadline1}
Assume, for contradiction that \(L\) is regular, then pumping lemma must hold.
Let \(p\) be the pumping length of \(L\), the \(dabc^pb^pb \in L\) because \(1
    \cdot p=p\).  Consider cases of selecting the middle substring \(y\) defined by
pumping lemma:
\begin{enumerate}
\item If \(y\) contains \(d\), then we cannot repeat it, since by definition \(d\)
       happens only once in the beginning of each word.
\item If \(y\) contains \(a\) we cannot repeat it, since in that case \(p \cdot x >
       p\) whenever \(x > 1\), which violates the definition of \(L\).
\item If we repeat any string containing \(c\), (but not containing \(b\), which
see in the next bullet point), we will violate the requirement that
\(\#_b(w) > \#_c(w) \cdot \#_a(w)\).
\item Since the prefix concatenated with \(y\) should not be longer than \(p\),
we cannot repeat any string containing \(b\) because the prefix of length
\(p\) has no \(b\) in it.
\end{enumerate}

Since these are our only options, it must be impossible to satisfy the
requirements of pumping lemma, hence, by contradiction, the language \(L\) is
not regular.

\subsection{Problem 2}
\label{sec:orgheadline4}
Prove that the language
\begin{align*}
  L = \{w \in \{x, y\}^* \;|\; (w = x^k(yyyy)^{m!},
  k,m \geq 2) \lor w = y^{2l}, l \geq 2 \}
\end{align*}

is not regular.

\subsubsection{Answer 2}
\label{sec:orgheadline3}
Suppose, for contradiction \(L\) is regular, then pumping lemma must be
applicable to it.  Consider the substrings of \(y^{2(p+1)}\), where \(p\) is the
pumping lenght.  Consider the second condition of pumping lemma, i.e. that
the prefix concatenated to the pumping substring should be no longer than
the pumping lenght, thus \(\abs{vw} \leq p\), this leaves us with the prefix
of at most \(y\) repeated \(p\) times, but if we repeat it, we will produce the
string \(y^{2(p+1)+p} = y^{2+3p}\).  Provided \(p > 2\), this string will be too
short to satisfy the requirements of the lemma.  Hence, the original
conjecture is falsified, hence \(L\) is not regualr.

\subsection{Problem 3}
\label{sec:orgheadline6}
Let \(L\) be a regular language over alphabet \(\Sigma\).  Prove that the following
languagage is also regular.

\begin{align*}
  Reversed(\mu_1 \mu_2 \dots \mu_n, n) = &\mu_1, \mu_2, \dots, \mu_n \in \Sigma, \\
                                         &\mu_1 \mu_2 \dots \mu_n \in L^R. \\
  Interleaved(\mu_1 \mu_2 \dots \mu_n, n) = &\exists (\sigma_1, \sigma_2, \dots, \sigma_n,
                                              \zeta_1, \zeta_2, \dots, \zeta_n \in \Sigma): \\
                                         &(\mu_1 \sigma_1 \zeta_1 \mu_2 \sigma_2 \zeta_2 
                                           \dots \mu_n \sigma_n \zeta_n \in L) \\
  \hat{L} = \{ &\mu_1 \mu_2 \dots \mu_n \;|\; \\
                                         &n \geq 0, \\
                                         &Reversed(\mu_1 \mu_2 \dots \mu_n, n), \\
                                         &Interleaved(\mu_1 \mu_2 \dots \mu_n, n) \}
\end{align*}

And whenever \(\epsilon \in L\), it is also the case that \(\epsilon \in \hat{L}\).

\subsubsection{Answer 3}
\label{sec:orgheadline5}
First I note that \(L^R\) is also regular, this is so because there must be a
DFA accepting \(L\) (by definition), we can transform this DFA in the following way:
\begin{enumerate}
\item Reverse all transitions.
\item Make the start state an accepting one.
\item Make all previously accepting states connect to a newly added start state by
\(\epsilon\)-transitions.  Thus we obtain an NFA for the reversed language.
\end{enumerate}

Thus, \(Rverersed(\dots)\) alone selects a regular language.

Next, I note that regular languages are closed under intersection.  Thus
proving that \(Interleaved(\dots)\) predicate selects a regular language will
prove that \(\hat{L}\) is regular, but we are given by definition, that
\(Interleaved(\dots)\) selects the language \(L\), or some regular subset of it,
hence \(\hat{L}\) must be regular.  The subset is regular because it is
essentially given to us by a regular expression \((\mu_n \sigma_n
    \zeta_n)^*\), where \(\mu_n\), \(\sigma_n\) and \(\zeta_n\) are character classes
of size at most \(n\).

\subsection{Problem 4}
\label{sec:orgheadline8}
Let \(L\) be a regular language over \(\Sigma\).  Prove that the following
language is also regular:
\begin{align*}
  \frac{1}{3}L = \{ w \in \Sigma^* \;|\; wxy \in L, \abs{x} = \abs{y} = \abs{w} \}
\end{align*}

\subsubsection{Answer 4}
\label{sec:orgheadline7}
The language \(\frac{1}{3}L\) is regular because it is possible to construct a
regular expression accepting it.  Since we know that regular languages are
closed under concatenation, we can devise a regular expression accepting the
\(w\) part of the language \(L\), and, similarly, for the \(xy\) part.  The
regular expression accepting the \(w\) part guarantees us that the language
\(\frac{1}{3}L\) is regular.

\subsection{Problem 5}
\label{sec:orgheadline12}
\begin{enumerate}
\item Write regular expression accepting the language \(0^*01^*/0^+\).
\item Prove that if \(L\) is regular then
\(\overset{\leftrightarrow}{L}=\{xy \in \Sigma^* \;|\; yx \in
      L\}\).
\item What is wrong with \(\overset{\leftrightarrow}{L} = (\Sigma^*
      \setminus L).(L / \Sigma^*)\) if it was offered as a solution for the
previous question?
\end{enumerate}

\subsubsection{Answer 5}
\label{sec:orgheadline9}
\(0^*01^*/0^+ = 0^*01^+\).  The rationale for this answer is that the string
in this language cannot end in 0, but the original regex would not accept
strings \(0^k\) where \(k < 2\), thus we have the result, where at least one
zero must be followed by at least one one.

\subsubsection{Answer 6}
\label{sec:orgheadline10}
The proof is immediate from concatenation closure properties: Languages of
\(x's\) and \(y's\) must be regular, because \(L\) is regular. Hence their
concatenation \(xy\) is regular too.

\subsubsection{Answer 7}
\label{sec:orgheadline11}
Assuming backwards slash means left quotient rather than complement, then
the general idea for the proof seems to be a workable one, except that one
shouldn't use the same character \(L\) to denote languages made of \(x's\) and
\(y's\).  To fix this, we could do the following:

\begin{align*}
  L_y &= L / \{ x \} \\
  L_x &= L \setminus \{ y \} \\
  \overset{\leftrightarrow}{L} &= (L \setminus L_y).(L / L_x)
\end{align*}
\end{document}