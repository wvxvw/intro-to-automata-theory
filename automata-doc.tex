% This LaTeX document was generated using the LaTeX backend of PlDoc,
% The SWI-Prolog documentation system



\subsection{automata.pl: High-level predicates for dealing with regular expressions}

\label{sec:automata}

\begin{tags}
    \tag{See also}
\url{https://github.com/wvxvw/intro-to-automata-theory}
\end{tags}

This module defines predicates for searching and replacing in strings
using regular expressions.\vspace{0.7cm}

\begin{description}
    \predicate[det]{match_regex}{2}{+Regex, +String}
Evaluates to true if \arg{String} is accepted by \arg{Regex}.

\begin{tags}
    \tag{See also}
\predref{match_suffix_regex}{3}, \predref{match_all_regex}{3}
\end{tags}

    \predicate[det]{match_suffix_regex}{3}{+Regex, +String, -Suffix}
Evaluates to true if \arg{Suffix} is the remaining part of the \arg{String}
not matched by \arg{Regex}.

\begin{tags}
    \tag{See also}
\predref{match_regex}{2}, \predref{match_all_regex}{3}
\end{tags}

    \predicate[nondet]{match_all_regex}{3}{+Regex, +String, -Match}
Instantiates \arg{Match} to all possible matches of \arg{Regex} in \arg{String}.

\begin{tags}
    \tag{See also}
\predref{match_regex}{2}, \predref{match_suffix_regex}{3}
\end{tags}

    \predicate[det]{invert_regex}{2}{+Straight, -Inverted}
\arg{Inverted} is a regular expression accepting the complement language
of the regular expression \arg{Straight}.
\end{description}

\subsection{automata(ast): Grammar constituents used when parsing regular expressions}

\label{sec:ast}

\begin{tags}
    \tag{See also}
\url{https://github.com/wvxvw/intro-to-automata-theory}
\end{tags}

This module defines predicates for generating abstract syntax trees
representing regular expressions.\vspace{0.7cm}

\begin{description}
    \predicate[nondet]{rterminal}{1}{?Regex}
Evaluates to true if \arg{Regex} is either an atom or an empty list.
Empty list denotes empty string, atoms stand for characters of
the strings.

\begin{tags}
    \tag{See also}
\predref{runion}{2}, \predref{rstar}{1}, \predref{rconcat}{2}, \predref{regex}{1}
\end{tags}

    \predicate[det]{runion}{2}{+Regex1, +Regex2}
Evaluates to true if \arg{Regex1} and \arg{Regex2} are valid regular expressions
as defined in \predref{regex}{1}.

\begin{tags}
    \tag{See also}
\predref{runion}{2}, \predref{rstar}{1}, \predref{rconcat}{2}, \predref{regex}{1}
\end{tags}

    \predicate[det]{rstar}{1}{+Regex}
Evaluates to true if \arg{Regex} is a valid regular expressions
as defined in \predref{regex}{1}.

\begin{tags}
    \tag{See also}
\predref{rterminal}{1}, \predref{rstar}{1}, \predref{rconcat}{2}, \predref{regex}{1}
\end{tags}

    \predicate[det]{rconcat}{2}{+Regex1, +Regex2}
Evaluates to true if \arg{Regex1} and \arg{Regex2} are valid regular expressions
as defined in \predref{regex}{1}.

\begin{tags}
    \tag{See also}
\predref{runion}{2}, \predref{rterminal}{1}, \predref{rconcat}{2}, \predref{regex}{1}
\end{tags}

    \predicate[det]{regex}{1}{+Regex}
Evaluates to true if \arg{Regex} is either a \predref{runion}{2}, or a \predref{rstar}{1}, or
a \predref{rconcat}{2} or a \predref{rterminal}{1}.

\begin{tags}
    \tag{See also}
\predref{runion}{2}, \predref{rstar}{1}, \predref{rconcat}{2}, \predref{rterminal}{1}
\end{tags}
\end{description}

\subsection{automata(convert): Convert between different automata represenations}

\label{sec:convert}

\begin{tags}
    \tag{See also}
\url{https://github.com/wvxvw/intro-to-automata-theory}
\end{tags}

This module defines defines conversions between regular expression
AST, DFA represented as a list of transitions or as a table, and NFA
represented similarly to DFA.

This module also defines data types:

\begin{itemize}
    \item Transition record

\begin{code}
trn(from:integer, to:integer, input:input, acc:boolean)
\end{code}

To describe a single transition between states on some input.
\const{acc} is \const{true} whenever the target state is an accepting one.
    \item State record

\begin{code}
state(label, acc:boolean)
\end{code}

To describe states.
    \item Transition table row record

\begin{code}
row(state:state, trns:trns)
\end{code}

To describe all inputs for a given state.
    \item Transition table record

\begin{code}
table(inputs:list(input), tab:list(row))
\end{code}

To describe a complete table of transitions between all states
of some automata.
\end{itemize}

\vspace{0.7cm}

\begin{description}
    \predicate[det]{has}{3}{+Accessor, ?Value, ?Record}
Flips arguments for Accessr (the predicate generated to access
fields of the record).

    \predicate[det]{regex_to_nfa}{2}{+Regex, -Nfa}
Evaluates to true when given regular expression \arg{Regex} can be
decomposed into a list of transitions \arg{Nfa}.

\begin{tags}
    \tag{See also}
\predref{gexps}{3}
\end{tags}

    \predicate[det]{nfa_inputs}{2}{+Nfa, -Inputs}
Evaluates to true when \arg{Inputs} is the alphabet of the \arg{Nfa} automata.

    \predicate[det]{nfa_states}{2}{+Nfa, -States}
Evaluates to true when \arg{States} is the states of the \arg{Nfa} automata.

    \predicate[det]{reacheable_states}{4}{+Input, +Transitions, +Table, -States}
Evaluates to true when \arg{States} can be reached from all \arg{Transitions}
on given \arg{Input}. This also accounts for epsilon transitions.

    \predicate[det]{nfa_table}{2}{+Transitions, -Table:table}
Evaluates to true when \arg{Table} is the transitions table containing
all transitions given by \arg{Transitions}.

    \predicate[det]{nfa_to_dfa}{2}{+Nfa, -Dfa}
Evaluates to true when \arg{Dfa} accepts the same language as \arg{Nfa}.

    \predicate[det]{table_to_diagram}{2}{+Table, -Diagram}
Evaluates to true when \arg{Diagram} contains all the transitions
described in \arg{Table}.

    \predicate[det]{dfa_to_regex}{2}{+Dfa, -Regex}
Evaluates to true when \arg{Regex} accepts the same language as the
given \arg{Dfa}. \arg{Dfa} could be either a transitions table or a list of
all transitions.

    \predicate[det]{dfa_minimize}{2}{+Dfa, -Minimized}
Evaluates to true when \arg{Minimized} is the minimal DFA of \arg{Dfa}.
\end{description}

\subsection{automata(parser): DCG rules for parsing regular expressions}

\label{sec:parser}

\begin{tags}
    \tag{See also}
\url{https://github.com/wvxvw/intro-to-automata-theory}
\end{tags}

This module defines DCG rules for parsing regular expressions from
string.\vspace{0.7cm}

\begin{description}
    \predicate[det]{gstar}{3}{+Exp, +Prefix, -Suffix}
Parses a regular expression followed by an asterisk (the Kleene
operator). Instantiates its first term to the rstar AST nonterminal.

\begin{tags}
    \tag{See also}
\predref{rstar}{1}
\end{tags}

    \predicate[det]{gunion}{3}{+Exp, +Prefix, -Suffix}
Parses a union of two regular expressions joined by the \verb$+$ sign.
Instantiates its first term to the runion AST nonterminal.

\begin{tags}
    \tag{See also}
\predref{runion}{2}
\end{tags}

    \predicate[det]{gchar}{3}{+Exp, +Prefix, -Suffix}
Parses a single terminal character and instantiates it to
AST rterminal term.

\begin{tags}
    \tag{See also}
\predref{rterminal}{1}
\end{tags}

    \predicate[det]{gexps}{3}{+Tree, +Prefix, -Suffix}
Parses regular expression from the string \arg{Prefix} and instantiates
the \arg{Tree} to the parse \predref{regex}{1} term.

\begin{tags}
    \tag{See also}
\predref{regex}{1}
\end{tags}
\end{description}

\subsection{automata(printing): Predicates for pretty printing}

\label{sec:printing}

\begin{tags}
    \tag{See also}
\url{https://github.com/wvxvw/intro-to-automata-theory}
\end{tags}

This module defines predicates useful to print structures generated by
other automata modules.\vspace{0.7cm}

\begin{description}
    \predicate[det]{regex_to_string}{2}{+Exp, -Result}
Evaluates to true when given regular expression \arg{Exp} can be written
as \arg{Result} string.

\begin{tags}
    \tag{See also}
\predref{regex}{1}
\end{tags}

    \predicate[det]{format_table}{1}{+Table:table}
Pretty-prints the contents of the transitions table.

\begin{tags}
    \tag{See also}
\predref{nfa_to_dfa}{2}, \predref{nfa_table}{2}
\end{tags}
\end{description}

\subsection{automata(utils): Vairous utilitis for working with lists}

\label{sec:utils}

\begin{tags}
    \tag{See also}
\url{https://github.com/wvxvw/intro-to-automata-theory}
\end{tags}

This module defines several utilities for working with lists
used across multiple other automata modules.\vspace{0.7cm}

\begin{description}
    \predicate[det]{reflexive_pairs}{2}{+Source, -Pairs}
Evaluates to true when \arg{Pairs} contains lists of length 2 each of
which contains a unique pair from the elements of \arg{Source}. For
this purpose \verb$[X, Y]$ is the same as \verb$[Y, X]$.

    \predicate[det]{cmp_lists}{3}{+List1, +List2, -Op}
Lexicographically compares \arg{List1} and \arg{List2}. \arg{Op} is set to either
one of \verb$<$, \verb$=$ and \verb$>$ respectively.

    \predicate[det]{replace_all}{4}{+Searched, +Replacement, +Source, -Destination}
Evaluates to true when \arg{Destination} contains all the elements of
\arg{Source} with \arg{Searched} element replaced by \arg{Replacement}.

\begin{tags}
    \tag{See also}
\predref{replace_all_lift}{4}, \predref{replace_all_tree}{4}, \predref{replace_all_tree_lift}{4}
\end{tags}

    \predicate[det]{replace_all_lift}{4}{+Replacement, +Searched, +Source, -Destination}
Same as \predref{replace_all}{4} except that \arg{Replacement} element is
substituted for each of the elements of \arg{Searched} list.

\begin{tags}
    \tag{See also}
\predref{replace_all}{4}, \predref{replace_all_tree}{4}, \predref{replace_all_tree_lift}{4}
\end{tags}

    \predicate[det]{replace_all_tree}{4}{+Searched, +Replacement, +Source, -Destination}
Same as \predref{replace_all}{4} except that \arg{Searched} cannot be a list, and
all sublists of \arg{Source} are processed recursively.

\begin{tags}
    \tag{See also}
\predref{replace_all}{4}, \predref{replace_all_lift}{4}, \predref{replace_all_tree_lift}{4}
\end{tags}

    \predicate[det]{replace_all_tree_lift}{4}{+Searched, +Replacement, +Source, -Destination}
Same as \predref{replace_all_tree}{4} except that \arg{Replacement} element is
substituted for each of the elements of \arg{Searched} list.

\begin{tags}
    \tag{See also}
\predref{replace_all}{4}, \predref{replace_all_tree}{4}, \predref{replace_all_lift}{4}
\end{tags}
\end{description}

