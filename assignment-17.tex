% Created 2016-01-29 Fri 22:35
\documentclass[11pt]{article}
\usepackage[utf8]{inputenc}
\usepackage[T1]{fontenc}
\usepackage{fixltx2e}
\usepackage{graphicx}
\usepackage{longtable}
\usepackage{float}
\usepackage{wrapfig}
\usepackage{rotating}
\usepackage[normalem]{ulem}
\usepackage{amsmath}
\usepackage{textcomp}
\usepackage{marvosym}
\usepackage{wasysym}
\usepackage{amssymb}
\usepackage{hyperref}
\tolerance=1000
\usepackage[utf8]{inputenc}
\usepackage{commath}
\usepackage{pgf}
\usepackage{tikz}
\usetikzlibrary{shapes,backgrounds}
\usepackage{marginnote}
\usepackage{listings}
\usepackage{enumerate}
\usepackage{algpseudocode}
\usepackage{algorithm}
\usepackage{mathtools}
\usetikzlibrary{arrows,automata}
\setlength{\parskip}{16pt plus 2pt minus 2pt}
\renewcommand{\arraystretch}{1.6}
\DeclareMathOperator{\Neg}{Neg}
\newcommand{\for}{\text{for }}
\author{Oleg Sivokon}
\date{\textit{<2016-01-23 Sat>}}
\title{Assignment 17, Authomata Theory}
\hypersetup{
  pdfkeywords={Automata Theory, Formal Languages, Assignment},
  pdfsubject={Seventh assignment in the course 20440 Automata and Formal Languages},
  pdfcreator={Emacs 24.5.1 (Org mode 8.2.10)}}
\begin{document}

\maketitle
\tableofcontents


\definecolor{codebg}{rgb}{0.96,0.99,0.8}
\definecolor{codestr}{rgb}{0.46,0.09,0.2}
\lstset{%
  backgroundcolor=\color{codebg},
  basicstyle=\ttfamily\scriptsize,
  breakatwhitespace=false,
  breaklines=false,
  captionpos=b,
  framexleftmargin=10pt,
  xleftmargin=10pt,
  framerule=0pt,
  frame=tb,
  keepspaces=true,
  keywordstyle=\color{blue},
  showspaces=false,
  showstringspaces=false,
  showtabs=false,
  stringstyle=\color{codestr},
  tabsize=2
}
\lstnewenvironment{maxima}{%
  \lstset{%
    backgroundcolor=\color{codebg},
    escapeinside={(*@}{@*)},
    aboveskip=20pt,
    captionpos=b,
    label=,
    caption=,
    showstringspaces=false,
    frame=single,
    framerule=0pt,
    basicstyle=\ttfamily\scriptsize,
    columns=fixed}}{}
}
\makeatletter
\newcommand{\verbatimfont}[1]{\renewcommand{\verbatim@font}{\ttfamily#1}}
\makeatother
\verbatimfont{\small}%
\clearpage

\section{Problems}
\label{sec-1}

\subsection{Problem 1}
\label{sec-1-1}
Prove that language $L = \{a^ib^{i+j}c^j \;|\; 1 \leq i \leq j\}$ is not
context-free.

\subsubsection{Answer 2}
\label{sec-1-1-1}
Suppose, for contradiction, $L$ is context-free, then, according to pumping
lemma, the following applies:
\begin{enumerate}
\item $p$ is the ``pumping length''.
\item For every word $z \in L$, $z = uvwxy$, s.t.
\item $\abs{vwx} \leq p$.
\item $\abs{vx} \geq 1$.
\item $uv^nwx^ny \in L$.
\end{enumerate}

Consider $p = i$, then there are five distinct ways to decompose $w$ into
$uvwxy$.  Of them three will decompose in a way such that both $v$ and $x$
are the same symbol, i.e. both $v$ and $x$ are either $a$, $b$ or $c$.

It is easy to see none of the above can be pumped: if $v = a^r$ and $x =
    a^s$ then by pumping $a$, eventually there will be more $as$ in $z$ than
$cs$, which contradicts $i \leq j$.  Similarly, if we pump $bs$, eventually
there will be more $bs$ than $as$ and $cs$ together.  Similarly for $cs$.

Another two possible decompositions are:
\begin{enumerate}
\item $v = a^r$ and $x = b^s$.  However, again, if we pump $as$, i.e. $r \neq
       0$, then eventually there will be more $as$ than $cs$.  And similarly for
$bs$.  When we pump $as$ and $bs$ together, eventually there will be more
$as$ than $cs$, again, contradicting $i \leq j$.
\item Thus the only case worth considering is where $v = b^r$ and $x = c^s$.
Consider the word $z = a^pb^{2p}c^p \in L$ with this decomposition.  If
either $r = 0$ or $s = 0$, we proceed as above, however, if $r = s \neq
       0$, then it must be the case that for all words $z' =
       a^pb^{p+p-r+r*i}c^{p-r+r*i}$, $z' \in L$. but it is not the case for
$i=0$.  Since $\abs{a^p} > \abs{c^{p-r}}$ contrary to the required $i
       \leq j$.
\end{enumerate}


These are all the possible decompositions of $z$, since neither can be
pumped, it must be the case that $L$ is not context-free.

\subsection{Problem 2}
\label{sec-1-2}
Prove that context-free languages are not closed under $max$ operation.

\subsubsection{Answer 2}
\label{sec-1-2-1}
Recall the definition of $max$:

\begin{align*}
  max(L) = \{u \in L \;|\; \forall v \in \Sigma^*: uv \in L \implies v = \epsilon\} \;.
\end{align*}

Let's take $L = \{a^nb^mc^k \;|\; n \leq k \lor m \leq k\}$.  $L$ is context-free,
since we can give a context-free grammar $L(G) = L$ as follows:

\begin{align*}
  &S \to X \;|\; Y \\
  &X \to aXC \;|\; C \\
  &C \to bCc \;|\; Cc \;|\; bBCc \;|\; c \\
  &B \to bB \;|\; b \\
  &Y \to AZ \\
  &A \to aA \;|\; \epsilon \\
  &Z \to bZc \;|\; Q \\
  &Q \to cQ \;|\; c \;.
\end{align*}

However, the $max(L) = \{a^nb^nc^n\}$, which is known to be non-context-free.

\subsection{Problem 3}
\label{sec-1-3}
Prove $L = \{a^ib^2c^j \;|\; i = 2j\}$ is constext free using closure
properties and some language from assignment 16.

\subsubsection{Answer 3}
\label{sec-1-3-1}
Recall that we proved language $M = \{a^kb^ic^jd^{j-i}e^k \;|\; 1 \leq i
    \leq j, k \geq 2\}$ to be context-free.  We can define homomorophism:

\begin{equation*}
  h(x) = \begin{cases}
    aa &\for x=a \\
    b &\for x=b \lor x=c \lor x=d \\
    c &\for x=e
  \end{cases} \;.
\end{equation*}

Now, $M' = h(M) = \{a^{2k}b^{i+j+j-i}c^k\}$, where $2j$ is any even integer,
thus could be rewritten as $\{a^{2k}b^{2j}c^k\}$.  Due to closure of
context-free languages under homomorphism, $M'$ is context-free.

Next, we can intersect $M'$ with a regular language $a^*b^2c^*$ to get $L$.
Since context-free languages are closed under intersection with regular
languages we proved that $L$ is context-free.

\subsection{Problem 4}
\label{sec-1-4}
Prove or disprove each of the following statements:
\begin{enumerate}
\item $L$ is a irregular context-free language.  $G$ is a context-sensitive
language.  $L \cap G$ is not context-free.
\item $L_1$ and $L_2$ are irregular context-free languages s.t. $L_1 \cap L_2
      \neq \emptyset$.  $L_1 \cap L_2$ is irregular context-free language.
\item $L$ is a regular language over $\Sigma$.  $G$ is a context-sensitive
language.  Define substitution $f$ s.t. $\forall \sigma \in \Sigma:
      f(\sigma) = G$.  $f(L)$ is context-sensitive.
\end{enumerate}

\subsubsection{Anwser 4}
\label{sec-1-4-1}
An interesection of a context-free and a context-sensitive languages may be
context-free.  For instance, $\{a^nb^n\} \cap \{a^nb^nc^n\} = \{a^nb^n\}$,
where $n \geq 0$ is context-free.

\subsubsection{Answer 5}
\label{sec-1-4-2}
An interesection of two context-free languages isn't necessarily irregular.
For instance $\{a^nb^n\} \cap \{a^nc^n\} = \{a^n\}$ where $n \geq 0$ is
regular.

\subsubsection{Answer 6}
\label{sec-1-4-3}
The language $L = \{\epsilon\}$ is regular.  $f(L) = L$ since no
substitution took place, hence this claim is false.

\subsection{Problem 5}
\label{sec-1-5}
Let $L$ be a context-free language over the alphabet $\Sigma = \{a,b,c,\dots
   z\}$.  Prove that $L'$ is also context-free, when defined as follows:

\begin{align*}
  L' = \{w \;|\; \abs{w} \equiv 0 \pmod{2}
                 \;\land\; \abs{w} \geq 4 
                 \;\land\; \textbf{Sub}(w)\}
\end{align*}


Where $\textbf{Sub}(w)$ is true whenever

\begin{equation*}
  w = \begin{cases}
    x\textbf{a}y\textbf{z}z &\for xpyqz \in L 
                             \land p \neq \textbf{a}
                             \land q \neq \textbf{z} \\

    x\textbf{z}y\textbf{a}z &\for xpyqz \in L 
                             \land p \neq \textbf{z}
                             \land q \neq \textbf{a}
  \end{cases} \\
  \abs{p} = \abs{q} = 1
\end{equation*}

\subsubsection{Answer 7}
\label{sec-1-5-1}
\begin{enumerate}
\item Provided $L$ is regular, we can bring its grammar $G$ to Greibach normal
form.
\item Now, for every rule of the form $A \to xA_1A_2A_3\dots A_n$ we introduce
new rules: $A \to aA'_1A'_2A'_3\dots A'_n$ whenever $x \neq a$ and 
$A' \to zA''_1A''_2A''_3\dots A''_n$ whenever $x \neq z$.
\item We replace the rules of the form $A \to x$ with $A'' \to x$.
\end{enumerate}


The resulting grammar $G'$ will nondeterministically substitute $a$ for some
terminal, which does not equal $a$ and $z$ for some terminal which does not
equal $z$.  It can only terminate the derivation when both substitutions
took place.  Using the same technique we can construct grammar $G''$ which
first replaces $z$ and then $a$.  The union of $G'$ and $G''$ (still a
context-free grammar, since context-free languages are closed under union)
will take care of $\textbf{Sub}(w)$ condition.

Now, we can take $G''' = (G' \cup G'') \cap R$, where $R = \{r \;|\; r \in
    \Sigma^+ \land \abs{r} \geq 4\}$.  Since $R$ is regular, and intersection of
context-free and regular languages is known to be context-free, $G'''$ must
be context-free.  This completes the proof.
% Emacs 24.5.1 (Org mode 8.2.10)
\end{document}
