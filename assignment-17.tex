% Created 2016-01-23 Sat 16:27
\documentclass[11pt]{article}
\usepackage[utf8]{inputenc}
\usepackage[T1]{fontenc}
\usepackage{fixltx2e}
\usepackage{graphicx}
\usepackage{longtable}
\usepackage{float}
\usepackage{wrapfig}
\usepackage{rotating}
\usepackage[normalem]{ulem}
\usepackage{amsmath}
\usepackage{textcomp}
\usepackage{marvosym}
\usepackage{wasysym}
\usepackage{amssymb}
\usepackage{hyperref}
\tolerance=1000
\usepackage[utf8]{inputenc}
\usepackage{commath}
\usepackage{pgf}
\usepackage{tikz}
\usetikzlibrary{shapes,backgrounds}
\usepackage{marginnote}
\usepackage{listings}
\usepackage{enumerate}
\usepackage{algpseudocode}
\usepackage{algorithm}
\usepackage{mathtools}
\usetikzlibrary{arrows,automata}
\setlength{\parskip}{16pt plus 2pt minus 2pt}
\renewcommand{\arraystretch}{1.6}
\DeclareMathOperator{\Neg}{Neg}
\author{Oleg Sivokon}
\date{\textit{<2016-01-23 Sat>}}
\title{Assignment 17, Authomata Theory}
\hypersetup{
  pdfkeywords={Automata Theory, Formal Languages, Assignment},
  pdfsubject={Seventh assignment in the course 20440 Automata and Formal Languages},
  pdfcreator={Emacs 24.5.1 (Org mode 8.2.10)}}
\begin{document}

\maketitle
\tableofcontents


\definecolor{codebg}{rgb}{0.96,0.99,0.8}
\definecolor{codestr}{rgb}{0.46,0.09,0.2}
\lstset{%
  backgroundcolor=\color{codebg},
  basicstyle=\ttfamily\scriptsize,
  breakatwhitespace=false,
  breaklines=false,
  captionpos=b,
  framexleftmargin=10pt,
  xleftmargin=10pt,
  framerule=0pt,
  frame=tb,
  keepspaces=true,
  keywordstyle=\color{blue},
  showspaces=false,
  showstringspaces=false,
  showtabs=false,
  stringstyle=\color{codestr},
  tabsize=2
}
\lstnewenvironment{maxima}{%
  \lstset{%
    backgroundcolor=\color{codebg},
    escapeinside={(*@}{@*)},
    aboveskip=20pt,
    captionpos=b,
    label=,
    caption=,
    showstringspaces=false,
    frame=single,
    framerule=0pt,
    basicstyle=\ttfamily\scriptsize,
    columns=fixed}}{}
}
\makeatletter
\newcommand{\verbatimfont}[1]{\renewcommand{\verbatim@font}{\ttfamily#1}}
\makeatother
\verbatimfont{\small}%
\clearpage

\section{Problems}
\label{sec-1}

\subsection{Problem 1}
\label{sec-1-1}
Prove that language $L = \{a^ib^{i+j}c^j \;|\; 1 \leq i \leq j\}$ is not
context-free.

\subsubsection{Answer 2}
\label{sec-1-1-1}
Suppose, for contradiction, $L$ is context-free, then, according to pumping
lemma, the following applies:
\begin{enumerate}
\item $p$ is the ``pumping length''.
\item For every word $z \in L$, $z = uvwxy$, s.t.
\item $\abs{vwx} \leq p$.
\item $\abs{vx} \geq 1$.
\item $uv^nwx^ny \in L$.
\end{enumerate}

Consider $p = i$, then there are five distinct ways to decompose $w$ into
$uvwxy$.  Of them three will decompose in a way such that both $v$ and $x$
are the same symbol, i.e. both $v$ and $x$ are either $a$, $b$ or $c$.

It is easy to see none of the above can be pumped: if $v = a^r$ and $x =
    a^s$ then by pumping $a$, eventually there will be more $as$ in $z$ than
$cs$, which contradicts $i \leq j$.  Similarly, if we pump $bs$, eventually
there will be more $bs$ than $as$ and $cs$ together.  Similarly for $cs$.

Another two possible decompositions are:
\begin{enumerate}
\item $v = a^r$ and $x = b^s$.  However, again, if we pump $as$, i.e. $r \neq
       0$, then eventually there will be more $as$ than $cs$.  And similarly for
$bs$.  When we pump $as$ and $bs$ together, eventually there will be more
$as$ than $cs$, again, contradicting $i \leq j$.
\item Thus the only case worth considering is where $v = b^r$ and $x = c^s$.
Consider the word $z = a^pb^{2p}c^p \in L$ with this decomposition.  If
either $r = 0$ or $s = 0$, we proceed as above, however, if $r = s \neq
       0$, then it must be the case that for all words $z' =
       a^pb^{p+p-r+r*i}c^{p-r+r*i}$, $z' \in L$. but it is not the case for
$i=0$.  Since $\abs{a^p} > \abs{c^{p-r}}$ contrary to the required $i
       \leq j$.
\end{enumerate}


These are all the possible decompositions of $z$, since neither can be
pumped, it must be the case that $L$ is not context-free.

\subsection{Problem 2}
\label{sec-1-2}
Prove that context-free languages are not closed under $max$ operation.

\subsubsection{Answer 2}
\label{sec-1-2-1}
Recall the definition of $max$:

\begin{align*}
  max(L) = \{u \in L \;|\; \forall v \in \Sigma^*: uv \in L \implies v = \epsilon\} \;.
\end{align*}

Let's take $L = \{a^nb^mc^k \;|\; n \leq k \lor m \leq k\}$.  $L$ is context-free,
since we can give a grammar $L(G) = L$ as follows:

\begin{align*}
  &S \to X \;|\; Y \\
  &X \to aXC \;|\; C \\
  &C \to bCc \;|\; Cc \;|\; bBCc \;|\; c \\
  &B \to bB \;|\; b \\
  &Y \to AZ \\
  &A \to aA \;|\; \epsilon \\
  &Z \to bZc \;|\; Q \\
  &Q \to cQ \;|\; c \;.
\end{align*}

However, the $max(L) = \{a^nb^nc^n\}$, which is known to be non-context-free.
% Emacs 24.5.1 (Org mode 8.2.10)
\end{document}
