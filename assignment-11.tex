% Created 2015-09-04 Fri 20:46
\documentclass[11pt]{article}
\usepackage[utf8]{inputenc}
\usepackage[T1]{fontenc}
\usepackage{fixltx2e}
\usepackage{graphicx}
\usepackage{longtable}
\usepackage{float}
\usepackage{wrapfig}
\usepackage{rotating}
\usepackage[normalem]{ulem}
\usepackage{amsmath}
\usepackage{textcomp}
\usepackage{marvosym}
\usepackage{wasysym}
\usepackage{amssymb}
\usepackage{capt-of}
\usepackage{hyperref}
\tolerance=1000
\usepackage[utf8]{inputenc}
\usepackage[usenames,dvipsnames]{color}
\usepackage{commath}
\usepackage{tikz}
\usetikzlibrary{shapes,backgrounds}
\usepackage{marginnote}
\usepackage{listings}
\usepackage{color}
\usepackage{enumerate}
\hypersetup{urlcolor=blue}
\hypersetup{colorlinks,urlcolor=blue}
\setlength{\parskip}{16pt plus 2pt minus 2pt}
\renewcommand{\arraystretch}{1.6}
\author{Oleg Sivokon}
\date{\textit{<2015-09-04 Fri>}}
\title{Assignment 11, Authomata Theory}
\hypersetup{
 pdfauthor={Oleg Sivokon},
 pdftitle={Assignment 11, Authomata Theory},
 pdfkeywords={Automata Theory, Formal Languages, Assignment},
 pdfsubject={First assignment in the course 20440 Automata and Formal Languages},
 pdfcreator={Emacs 25.0.50.1 (Org mode 8.3beta)}, 
 pdflang={English}}
\begin{document}

\maketitle
\tableofcontents

\definecolor{codebg}{rgb}{0.96,0.99,0.8}
\definecolor{codestr}{rgb}{0.46,0.09,0.2}
\lstset{%
  backgroundcolor=\color{codebg},
  basicstyle=\ttfamily\scriptsize,
  breakatwhitespace=false,
  breaklines=false,
  captionpos=b,
  framexleftmargin=10pt,
  xleftmargin=10pt,
  framerule=0pt,
  frame=tb,
  keepspaces=true,
  keywordstyle=\color{blue},
  showspaces=false,
  showstringspaces=false,
  showtabs=false,
  stringstyle=\color{codestr},
  tabsize=2
}
\lstnewenvironment{maxima}{%
  \lstset{%
    backgroundcolor=\color{codebg},
    escapeinside={(*@}{@*)},
    aboveskip=20pt,
    captionpos=b,
    label=,
    caption=,
    showstringspaces=false,
    frame=single,
    framerule=0pt,
    basicstyle=\ttfamily\scriptsize,
    columns=fixed}}{}
}
\makeatletter
\newcommand{\verbatimfont}[1]{\renewcommand{\verbatim@font}{\ttfamily#1}}
\makeatother
\verbatimfont{\small}%
\clearpage

\section{Problems}
\label{sec:orgheadline10}

\subsection{Problem 1}
\label{sec:orgheadline3}
Given following languages over the alphabet \(\{a, b\}\)
\begin{itemize}
\item \(L_1 = \emptyset\).
\item \(L_2 = \{\epsilon, aa\}\).
\item \(L_3 = \{\epsilon, a, aa, ab, abb\}\).
\item \(L_4 = \{aabb, aabbb, aa, aaa\}\).
\item \(L_5 = \{\epsilon, b, bbb, abab, abba, aabb\}\).
\item \(L_6 = \{\epsilon, bbbaa, baba, aaab, aabba, aa\}\).
\end{itemize}


\begin{enumerate}
\item What are the following languages:
\begin{itemize}
\item \(L_4L_4\).
\item \((L_1 \cup L_2 \cup L_3)^R\).
\item \(L_3L_1L_6\).
\end{itemize}

\item Define exponentiation as follows:
\(L^K = \{x \in L \;|\; \exists y \in K.(\abs{y} = \abs{x})\}\).
What are the languages \(L_4^{L_5}\) and \(L_6^{L_1}\).
\end{enumerate}


\lstset{language=prolog,label= ,caption= ,captionpos=b,numbers=none}
\begin{lstlisting}
:- use_module(library(lists)).

concatentated_member(L1, L2, L3) :-
    member(M1, L1), member(M2, L2),
    string_concat(M1, M2, L3).

concatentated(L1, L2, L3) :-
    findall(X, concatentated_member(L1, L2, X), X),
    list_to_set(X, L3).

assignment_11a :-
    X = ["aabb", "aabbb", "aa", "aaa"],
    concatentated(X, X, Y),
    [First | Rest] = Y,
    write("$$\\{"),
    write(First),
    maplist(format(',\\allowbreak ~s'), Rest),
    write("\\}$$").
\end{lstlisting}

\(\{aabbaabb,\allowbreak aabbaabbb,\allowbreak aabbaa,\allowbreak aabbaaa,\allowbreak aabbbaabb,\allowbreak aabbbaabbb,\allowbreak aabbbaa,\allowbreak aabbbaaa,\allowbreak aaaabb,\allowbreak aaaabbb,\allowbreak aaaa,\allowbreak aaaaa,\allowbreak aaaaabb,\allowbreak aaaaabbb,\allowbreak aaaaaa\}\)

\subsubsection{Answer 1}
\label{sec:orgheadline1}
\begin{enumerate}
\item Concatenation of \(L_4\) with itself gives:
\(L_4L_4 = \{aabbaabb, aabbaabbb, aabbaa,\;\) \(aabbaaa, aabbbaabb,
       aabbbaa, aabbbaaa, aaaabbb, aaaa, aaaaa, aaaaabb, aaaaabbb\}\)
\item \((L_1 \cup L_2 \cup L_3)^R = \{\epsilon, a, aa, ba, bba\}\).
\item \(L_3L_1L_6 = \emptyset\).  This is so because there are no words
in language \(L_1\) to concatenate with.
\end{enumerate}

\subsubsection{Answer 2}
\label{sec:orgheadline2}
\begin{enumerate}
\item \(L_4^{L_5} = \{aaa, aabb\}\).
\item \(L_6^{L_1} = \emptyset\).
\end{enumerate}

\subsection{Problem 2}
\label{sec:orgheadline6}
Let \(L_1, L_2\) and \(L_3\) be languages over some alphabet \(\Sigma\).
Prove or disprove:
\begin{enumerate}
\item \((L_1 \cup L_2) L_3 = L_1 L_3 \cup L_2 L_3\).
\item \((L_1 \cap L_2) L_3 = L_1 L_3 \cap L_2 L_3\).
\end{enumerate}

\subsubsection{Answer 3}
\label{sec:orgheadline4}
First, I will prove \((L_1 \cup L_2) L_3 \subset L_1 L_3 \cup L_2 L_3\).
Assume to the contrary that there is \(w \in (L_1 \cup L_2) L_3\) which is not
in \(L_1 L_3 \cup L_2 L_3\).  Put \(w = xy\) where \(x \in (L_1 \cup L_2)\) and \(y
    \in L_3\) (this implies \(L_3 \neq \emptyset\) and at least one of \((L_1 \cup
    L_2) \neq \emptyset\).  Suppose \(x\) comes from \(L_1\), then it has to be in
\(L_1 L_3 \cup L_2 L_3\) because it is in L\(_{\text{1}}\) L\(_{\text{3}}\)\$, similartly if it originates
in \(L_2\).  Suppose now \(L_3 = \emptyset\), then there is an empty set on
both sides of equation (by definition of concatenation).  Suppose both \(L_1\)
and \(L_2\) are empty, then, again, we have emtpy set on both sides of the
equation.  Thus we showed that it is impossible for \(w\) not to be in the
\(L_1 L_3 \cup L_2 L_3\), hence the original argument must be true.

Similarly, to prove \(L_1 L_3 \cup L_2 L_3 \subset (L_1 \cup L_2) L_3\),
assume there exists \(w \in L_1 L_3 \cup L_2 L_3\), not a amember of \((L_1
    \cup L_2) L_3\).  Again, \(w = xy\) where \(y \in L_3\) and \(x\) may be a
member of \(L_1\), \(L_2\) or both.  Suppose, again, the sets aren't empty.
If \(w\) came from \(L_1 L_3\), then \(x\) came from \(L_1\), but it is a member
of \((L_1 \cup L_2)\) and similarly if it came from \(L_2\).  Since \(y \in L_3\)
and \(L_3\) is present on both sides, it is not possible for \(w\) to not
be a member of \((L_1 \cup L_2) L_3\).  As in previous case, whenever \(L_3\)
or \((L_1 \cup L_2)\) are empty, both sides of equation contain an empty set.
Hence we proved both directions, hence the conjecture is true.

\subsubsection{Anser 4}
\label{sec:orgheadline5}
This conjecture isn't generally true.  Suppose \(L_1 = L_3 = \{a\}\) and
\(L_3 = \{\epsilon, aa\}\).  Then:

\begin{enumerate}
\item \((L_1 \cap L_2) L_3 = \emptyset\).
\item \(L_1 L_3 \cap L_2 L_3 = \{aa\}\).
\end{enumerate}

I.e. both sides of equation are not equal.  This completes the proof.

\subsection{Problem 3}
\label{sec:orgheadline9}
An equivalence relation over \(\Sigma^*\) will be called invariant from
right if \(\forall z \in \Sigma^*.(xRy \implies xzRyz)\).  Answer for
every relation in \(\{a, b\}^*\) whether the relation is an equivalence
relation and whether it is invariant from right.

\begin{enumerate}
\item \(xRy \iff \abs{x} \geq \abs{y}\).
\item \(xRy \iff (\abs{x} = \abs{y} = 0 \lor x = qz, y = pz, \abs{z} \geq 1)\).
\end{enumerate}

\subsubsection{Answer 5}
\label{sec:orgheadline7}
Total order relation is not symmetric.  Suppose \(x = a\) and \(y = ab\), then
\(x \geq y\) but not \(y \geq x\).  Since this relation is not an equivalence,
it cannot be right invariant either.

\subsubsection{Answer 6}
\label{sec:orgheadline8}
This relation is an equivalence.  It is transitive because whenever
\(x = qz\), \(y = pz\) and \(w = vz\), all of the below hold: \(xRy\), \(yRw\),
\(xRw\) since they all have the last letter in common.  This also holds
trivially in case the length is zero, since \(x = y = w = \epsilon\) in
that case.

The relation is reflexive because whenever every string is either
empty or its last symbol is equal to itself, i.e. \(xRx\) is always true.

The relation is symmetric because whenever \(x = qz\) and \(y = pz\) then
both \(xRy\) and \(yRx\) hold (again, becuase \(x\) and \(y\) have the final
letter in common, or are both the empty string).

The relation is also invariant from the right.  The proof will proceed
by induction on the string's length.

\textbf{Base step:} \(\epsilon R \epsilon \implies \epsilon z R \epsilon z\) because
\(R\) is reflexive and \(z = \epsilon z\).

\textbf{Inductive step:} suppose the inductive hypotesis \(xRy \implies xzRyz\), then
suppose we concatenate the same character \(c\) to both \(x\) and \(y\).  This
character must be the same by definition of \(R\).  Then \(xcRyc \implies
    xczRycz\) because we can simply rename \(xc = x_1\) adn \(yc = y_1\) and obtain
the inductive hypothesis restated using new terms: \(x_1Ry_1 \implies
    x_1zRy_1z\).  This completes the inductive step, and hence the proof is
completed.
\end{document}